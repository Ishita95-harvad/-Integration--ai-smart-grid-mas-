
\documentclass[12pt]{report}
\usepackage[utf8]{inputenc}
\usepackage{graphicx}
\usepackage{amsmath}
\usepackage{longtable}
\usepackage{geometry}
\usepackage{booktabs}
\geometry{a4paper, margin=1in}

\title{Chapter 3: GridLAB-D Simulation Setup}
\author{}
\date{\today}

\begin{document}

\maketitle

\chapter*{Chapter 3: GridLAB-D Simulation Setup}

To validate and assess the effectiveness of the Multi-Agent System (MAS) designed for energy efficiency and climate-responsive operations, simulations were conducted using GridLAB-D, an open-source simulation platform developed by the U.S. Department of Energy for modeling electric power distribution systems. GridLAB-D supports granular modeling of agents, distributed energy resources (DERs), load dynamics, and control strategies, making it suitable for smart grid research.

\section*{3.1 Simulation Design}

\subsection*{3.1.1 Microgrid Topology}
The virtual grid comprises:
\begin{itemize}
    \item Residential loads (low voltage, high variability)
    \item Commercial buildings (moderate voltage, peak daytime consumption)
    \item Industrial nodes (high voltage, constant base loads)
\end{itemize}

\textbf{Distributed Energy Resources (DERs):}
\begin{itemize}
    \item Solar PV panels (rooftop and ground-mounted)
    \item Battery Energy Storage Systems (BESS) with two-way inverters
    \item Wind turbines placed in suitable commercial zones
\end{itemize}

All DER nodes are connected through feeder lines modeled with impedance data, transformer losses, and bus voltage constraints. The IEEE-13 node and modified IEEE-123 node systems were used as base topologies.

\subsection*{3.1.2 Role of Agents in the Topology}
Agents are logically and computationally embedded within specific grid nodes:
\begin{itemize}
    \item \textbf{Forecast Agent:} Predicts demand and renewable generation using LSTM models and ARIMA baselines.
    \item \textbf{Load Agent:} Applies DR mechanisms to shift HVAC and EV usage.
    \item \textbf{Renewable Agent:} Manages solar/battery behavior and limits reverse power flow.
    \item \textbf{Coordinator Agent:} Handles inter-agent messaging and KPI tracking.
\end{itemize}

\subsection*{3.1.3 Simulation Timeframe and Resolution}
Six-month simulation (Jan–Jun) with 15-minute resolution for fine-grained control and learning cycles.

\section*{3.2 Data Inputs}

\subsection*{3.2.1 Meteorological Data}
\begin{itemize}
    \item Solar radiation and temperature from IMD
    \item Wind speed from MNRE, normalized using power-law
\end{itemize}

\subsection*{3.2.2 Load Profiles}
\begin{itemize}
    \item Residential and commercial profiles from CEA and POSOCO
    \item Industrial base load simulations
    \item Appliance modeling (HVAC, EV, lighting)
\end{itemize}

\subsection*{3.2.3 Generation Profiles}
\begin{itemize}
    \item Solar data from NREL
    \item Wind from MNRE scaled by turbine curves
    \item Battery SoC, efficiency, degradation
\end{itemize}

\section*{3.3 Agent Behavior and Learning Cycle}

\subsection*{3.3.1 Data Acquisition}
\begin{itemize}
    \item Node-level voltage/current
    \item Forecasted vs. actual data
    \item SoC tracking
\end{itemize}

\subsection*{3.3.2 Decision Making}
Q-learning-based agents choose:
\begin{itemize}
    \item Load shifting
    \item Battery dispatch
    \item HVAC/lights setpoints
\end{itemize}
Rewards based on stability, savings, and renewable use.

\subsection*{3.3.3 Coordination}
ZeroMQ-based message passing includes:
\begin{itemize}
    \item Forecasts
    \item DR signals
    \item Rewards
\end{itemize}

\subsection*{3.3.4 Learning and Adaptation}
Q-table and policy updates driven by delayed rewards.

\section*{3.4 Scenario Descriptions}
\begin{itemize}
    \item \textbf{Scenario 1:} No MAS, static DERs
    \item \textbf{Scenario 2:} MAS, internal feedback only
    \item \textbf{Scenario 3:} MAS with weather-aware agents
\end{itemize}

\section*{3.5 Performance Metrics}

\begin{longtable}{@{}lll@{}}
\toprule
\textbf{KPI} & \textbf{Definition} & \textbf{Goal} \\
\midrule
Energy Efficiency & Delivered / Consumed & Maximize \\
Renewable Utilization & \% of demand met & $>$60\% \\
Peak Demand Reduction & Peak vs. Avg. load & Minimize \\
Forecast Accuracy & RMSE, MAE & RMSE $<$ 2 kW \\
Loss Minimization & Line + conversion losses & $<$10\% \\
Agent Responsiveness & Mean delay & $<$1 min \\
\bottomrule
\end{longtable}

\section*{3.6 Results Summary}

\begin{longtable}{@{}llll@{}}
\toprule
\textbf{Metric} & \textbf{Scenario 1} & \textbf{Scenario 2} & \textbf{Scenario 3} \\
\midrule
Energy Efficiency (\%) & 72.5 & 81.4 & 90.6 \\
Renewable Utilization (\%) & 38.7 & 56.2 & 65.3 \\
Peak Demand Reduction (\%) & 4.8 & 17.2 & 24.1 \\
Forecast RMSE (kW) & 3.2 & 2.4 & 1.7 \\
Forecast MAE (kW) & 2.1 & 1.8 & 1.1 \\
\bottomrule
\end{longtable}

\section*{3.7 Visualization}
Streamlit and Matplotlib Dashboard:
\begin{itemize}
    \item Time slider for curve visualization
    \item Toggle between scenarios
    \item Battery SoC and Q-table heatmaps
    \item RMSE convergence graphs
\end{itemize}

\end{document}
